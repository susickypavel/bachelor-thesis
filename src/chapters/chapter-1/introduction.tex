\chapter{Úvod}

Přístupnost na webu je v dnešní době důležitá. Tato práce se zabývá implementací JavaScriptové knihovny
znovupoužitelných, nestylovaných (low-level) komponent, které jsou implementované podle moderních specifikací publikovaných konsorciem W3C.

V teoretické části se věnuji problematice přístupnosti na webu, kde popisuji účel existence
Web Accessibility Initiative --- Accessible Rich Internet Applications (dále jen \textbf{WAI-ARIA}).
Součástí této kapitoly je analýza existujících doporučení pro tvorbu webových
komponent pod názvem ARIA Authoring Practices Guide (dále jen \textbf{APG}) a
obecných doporučení pro přístupný web Web Content Accessibility Guidelines (dále jen \textbf{WCAG}).

V praktické části práce se zabývám implementací knihovny komponent,
které jsou přístupné, tedy jsou implementované podle doporučení od WAI-ARIA.

\section{Motivace}

V současné době jsou znovupoužitelné komponenty nejčastějším způsobem jakým se vytváří webové aplikace, protože většina knihoven (například React, Vue, Angular, SolidJS a další) využívají komponentové paradigma.

Vznik komponent tak vedl k vytvoření souboru komponent tzv.\ component libraries obsahující základní prvky, které denně potkáváme na internetu, aby je vývojář nemusel znovu implementovat pro každou aplikaci zvlášť.

Vývoj znovupoužitelných komponent je však náročný, protože kromě implementačních detailů je nutné se zabývat i přístupové problematice. Toto vedlo k nekonzistenci napříč knihovnami, kde nejvíce používané knihovny (React, Vue, \dots) takové knihovny komponent mají, ale méně používané knihovny (Svelte, SolidJS, Lit) takové knihovny nemají, nebo jejich kvalita je nízká.

\section{Cíl práce}


Hlavním výsledkem této práce by měla být knihovna, která rozšíří ekosystém frameworku Svelte o znovupoužitelné, stylovatelné a přístupné komponenty.


\begin{figure}
    \centering
    \includegraphics[width=0.5\textwidth]{./assets/figures/chapter-1/otter.jpg}
    \caption{Otters are awesome!}
\end{figure}
