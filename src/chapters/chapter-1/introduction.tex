\chapter{Úvod}

Přístupnost na webu je v dnešní době důležitá. Tato práce se zabývá implementací JavaScriptové knihovny
znovupoužitelných, nestylovaných (low-level) komponent, které jsou implementované podle moderních specifikací publikovaných od W3C konsorcia.

V teoretické části se věnuji problematice přístupnosti na webu, kde popisuji účel existence
Web Accessibility Initiative --- Accessible Rich Internet Applications (dále již \textbf{WAI-ARIA}).
Součástí této kapitoly je analýza existujících doporučení pro tvorbu elementárních
komponent pod názvem ARIA Authoring Practices Guide (dále již \textbf{APG}) a
obecných doporučení stanovených Web Content Accessibility Guidelines (dále již \textbf{WCAG}).

V praktické části práce se zabývám implementací knihovny komponent,
které jsou přístupné, tedy jsou implementované podle doporučení od WAI-ARIA.

\section{Motivace}

TODO

\section{Cíl práce}


\begin{figure}
    \centering
    \includegraphics[width=0.5\textwidth]{./assets/figures/chapter-1/otter.jpg}
    \caption{Otters are awesome!}
\end{figure}
