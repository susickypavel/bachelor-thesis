\chapter{Úvod}

Přístupnost na webu je v dnešní době důležitá. Tato práce se zabývá implementací JavaScriptové knihovny
znovupoužitelných, nestylovaných (headless) komponent, které jsou implementované podle moderních specifikací publikovaných konsorciem W3C.

V teoretické části se věnuji problematice přístupnosti na webu, kde popisuji účel existence \textbf{\gls{waiaria}}.
Součástí této kapitoly je analýza existujících doporučení pro tvorbu webových
komponent pod názvem \textbf{\gls{apg}} a obecných doporučení pro přístupný web \textbf{\gls{wcag}}.

V praktické části práce se zabývám implementací knihovny komponent,
které jsou přístupné, tedy jsou implementované podle standardů a doporučení od \gls{wai}.

\section{Motivace}

V současné době jsou znovupoužitelné komponenty nejčastějším způsobem jakým se vytváří webové aplikace, protože nejpoužívanější knihovny využívají komponentové paradigma~\cite{react,vue,solid,svelte}.

Vznik komponent tak vedl k vytvoření knihoven tzv.\ component libraries obsahující základní prvky, které denně potkáváme na internetu, aby je vývojář nemusel znovu implementovat pro každou aplikaci zvlášť.

Vývoj znovupoužitelných komponent je však náročný, protože kromě implementačních detailů je nutné se zabývat i přístupové problematice.

Toto vedlo k nekonzistenci ekosystémů, kde hojně využívané knihovny mají k dispozici knihovny se základními i robustními, přístupnými komponenty na webu. Oproti tomu menší knihovny, které se teprve začínají dostávat do podvědomí vývojářům, mají k dispozici zásadně menší ekosystém komponent, který není zdaleka tak robustný.

\section{Cíl práce}

Hlavním výsledkem této práce by měla být knihovna, která rozšíří ekosystém frameworku Svelte o znovupoužitelné, stylovatelné a přístupné komponenty implementované dle návrhových vzorů APG.

Vývojáři budou schopni využít tyto komponenty ve vlastních projektech a ušetří jim tak čas strávený na vývoji přístupných webových aplikací.
