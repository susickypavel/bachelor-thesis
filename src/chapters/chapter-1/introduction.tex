\chapter{Úvod}

Přístupnost na webu je v dnešní době důležitá. Tato práce se zabývá implementací JavaScriptové knihovny
znovupoužitelných a přístupných komponent, které jsou implementované podle moderních specifikací publikovaných konsorciem W3C.

V teoretické části se rozebírá problematika přístupnosti na webu, kde se popisuje účel existence \textbf{\gls{wai}}.
Součástí této kapitoly je analýza technických specifikací a doporučení pro tvorbu přístupného webového obsahu
pod názvy \textbf{\gls{waiaria}}, obecných doporučení pro obsah na webu \textbf{\gls{wcag}} a doporučení pro tvorbu přístupných komponent \textbf{\gls{apg}}.

Praktická část se věnuje popisu, návrhu a implementaci knihovny komponent,
která je přístupná. Komponenty jsou tedy implementované podle standardů a doporučení zmíněných výše.

\section{Motivace}

V současné době jsou znovupoužitelné komponenty nejčastějším způsobem jakým se vytváří webové aplikace, protože nejpoužívanější knihovny využívají komponentové paradigma~\cite{react,vue,solid,svelte}.

Vznik komponent tak vedl k vytvoření knihoven komponent obsahující základní prvky, které denně potkáváme na internetu, aby je vývojář nemusel znovu implementovat pro každou aplikaci zvlášť.

Vývoj znovupoužitelných komponent je však náročný, protože kromě implementačních detailů je nutné se zabývat i přístupové problematice.

Toto vedlo k nekonzistenci ekosystémů, kde hojně využívané knihovny mají k dispozici knihovny se základními i robustními, přístupnými komponenty na webu. Oproti tomu menší knihovny, které se teprve začínají dostávat do podvědomí vývojářům, mají k dispozici zásadně menší ekosystém komponent, který není zdaleka tak robustný.

\section{Cíl práce}

Hlavním výsledkem této práce by měla být knihovna, která rozšíří ekosystém frameworku Solid.js o znovupoužitelné a přístupné komponenty implementované dle návrhových vzorů APG.

Vývojáři budou schopni využít tyto komponenty ve vlastních projektech a ušetří jim tak čas strávený na vývoji přístupných webových aplikací.

Součástí výstupu je i webová stránka dokumentující použití komponent, vygenerovaná dokumentace pomocí typedocu, automatizované testy a příkladová aplikace využívající dané komponenty.
