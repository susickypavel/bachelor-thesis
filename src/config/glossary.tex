\usepackage[style=altlist]{glossaries}

\makenoidxglossaries{}

\newacronym{wcag}{WCAG}{Web Content Accessibility Guidelines}
\newacronym{waiaria}{WAI-ARIA}{Web Accessibility Initiative --- Accessible Rich Internet Applications}
\newacronym{apg}{APG}{ARIA Authoring Practices Guide}
\newacronym{w3c}{W3C}{World Wide Web Consortium}
\newacronym{wai}{WAI}{Web Accessibility Initiative}
\newacronym{cli}{CLI}{Command Line Interface}
\newacronym{hta}{HTA}{Hierarchical Task Analysis}
\newacronym{ide}{IDE}{Integrated Development Environment}
\newacronym{api}{API}{Application Programming Interface}
\newacronym{e2e}{E2E}{End-to-End}

\newacronym{dom}{DOM}{Document Object Model}
\newacronym{vdom}{VDOM}{Virtual DOM}

\newglossaryentry{lighthouse}{
    name={Ligthouse},
    description={Lighthouse je nástroj od Google pro provádění auditu stránky z různých hledisek. V kontextu této práce je využivána ``Time To Interactive'' metrika a velikost přenesených dat},
}

\newglossaryentry{todomvc}{
    name={TodoMVC},
    description={TodoMVC je kolekce příkladové aplikace implementované v různých webových technologiích},
}

\newglossaryentry{jsx}{
    name={JSX},
    description={JSX je syntaxe pro psaní HTML syntaxe v JavaScriptu}
}
