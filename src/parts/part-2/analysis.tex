\chapter{Analýza existujících řešení}

Tato kapitola rozebírá existující řešení v rámci JavaScriptového ekosystému.
Nejdříve rozeberu existující řešení pro React, protože hodně knihoven ve Solid.js z nich vychází.
Následně se podívám na Solid.js a rozeberu chybějící části v rámci jeho ekosystému.

\section{React}

\subsection{react-aria}

React aria je open-source projekt od společnosti Adobe, který obsahuje velice robustní sadu komponent a primitiv v podobě React hooks.
Obsahuje velké množství komponent a interakcí, od nejzákladnějších jako jsou tlačítka, formulářové prvky až po složitější komponenty jako kalendáře.
Knihovna je velice dobře dokumentovaná a podporována komunitou.
Zároveň je distribuovaná bez stylů, tedy se hodí pro využití v rámci designových systémů.

\begin{table}[ht]
    \begin{ctucolortab}
        \begin{tabularx}{\textwidth}{Y Y}
            \bfseries \textcolor{OK}{Výhody} & \bfseries \textcolor{NOT_OK}{Nevýhody} \\\Midrule{}
            Flexibilita použití              & Větší úsilí na udržování               \\
            Rozmanitost komponent
        \end{tabularx}
    \end{ctucolortab}
    \caption{Shrnutí výhod a nevýhod knihovny react-aria}
\end{table}

% \subsection{Radix UI}

% TODO

% \begin{table}[ht]
%     \begin{ctucolortab}
%         \begin{tabularx}{\textwidth}{Y Y}
%             \bfseries \textcolor{OK}{Výhody} & \bfseries \textcolor{NOT_OK}{Nevýhody} \\\Midrule{}
%             TODO                             & TODO                                   \\
%             TODO                             & TODO
%         \end{tabularx}
%     \end{ctucolortab}
%     \caption{Shrnutí výhod a nevýhod knihovny Radix UI}
% \end{table}

% \subsection{Shadcn UI}

% TODO

% \begin{table}[ht]
%     \begin{ctucolortab}
%         \begin{tabularx}{\textwidth}{Y Y}
%             \bfseries \textcolor{OK}{Výhody} & \bfseries \textcolor{NOT_OK}{Nevýhody} \\\Midrule{}
%             TODO                             & TODO                                   \\
%             TODO                             & TODO
%         \end{tabularx}
%     \end{ctucolortab}
%     \caption{Shrnutí výhod a nevýhod Shadcn UI}
% \end{table}

% \subsection{Headless UI}

% TODO

% \begin{table}[ht]
%     \begin{ctucolortab}
%         \begin{tabularx}{\textwidth}{Y Y}
%             \bfseries \textcolor{OK}{Výhody} & \bfseries \textcolor{NOT_OK}{Nevýhody} \\\Midrule{}
%             TODO                             & TODO                                   \\
%             TODO                             & TODO
%         \end{tabularx}
%     \end{ctucolortab}
%     \caption{Shrnutí výhod a nevýhod knihovny Headless UI}
% \end{table}

\clearpage

\section{Svelte}

\subsection{Melt UI}

Podobně jako react-aria je Melt UI knihovna primitiv, v kontextu této knihovny se primitiva nazývají ``builders''.
Uživatelé konzumují tyto primitiva vytváří si tak komponenty s vlastní HTML strukturou.
Z toho plyne, že je knihovna velice flexibilní na používání, ale zároveň je tak zvýšená náročnost na udržování komponent.
Výhodou je, že je možnost zde vytvořit příkladové použití těchto primitiv včetně základních stylů a usnadnit tak konzumentům práci.

\begin{table}[ht]
    \begin{ctucolortab}
        \begin{tabularx}{\textwidth}{Y Y}
            \bfseries \textcolor{OK}{Výhody} & \bfseries \textcolor{NOT_OK}{Nevýhody} \\\Midrule{}
            Vysoká flexibilita použití       & Vyšší nároky na udržování
        \end{tabularx}
    \end{ctucolortab}
    \caption{Shrnutí výhod a nevýhod knihovny Melt UI}
\end{table}

% \subsection{Svelte Headless UI}

% TODO

% \begin{table}[ht]
%     \begin{ctucolortab}
%         \begin{tabularx}{\textwidth}{Y Y}
%             \bfseries \textcolor{OK}{Výhody} & \bfseries \textcolor{NOT_OK}{Nevýhody} \\\Midrule{}
%             TODO                             & TODO                                   \\
%             TODO                             & TODO
%         \end{tabularx}
%     \end{ctucolortab}
%     \caption{Shrnutí výhod a nevýhod knihovny Svelte Headless UI}
% \end{table}
