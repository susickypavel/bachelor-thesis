\chapter{Rešerše existujících řešení}

Tato kapitola rozebírá existující řešení v rámci JavaScriptového ekosystému.
Vyskytuje se tu bližší pohled na existující řešení pro React a Svelte protože hodně knihoven ve Solid.js z nich vychází.
Následně se rozebírá Solid.js a jeho chybějící části v rámci ekosystému.

\section{React}

React má obrovské zastoupení v rámci JavaScriptového ekosystému, proto kvalita knihoven zde bude vysoká díky velkému množství vývojářů a času věnovanému vývoji v této oblasti.

\subsection{React Aria}

React Aria\footnote{\url{https://react-spectrum.adobe.com/react-aria}} je open-source projekt od společnosti Adobe, který obsahuje velice robustní sadu primitiv v podobě React hooks.
Obsahuje velké množství komponent a interakcí, od nejzákladnějších jako jsou tlačítka, formulářové prvky až po složitější komponenty jako kalendáře.
Knihovna je velice dobře dokumentovaná a podporována komunitou.
Zároveň je distribuovaná bez stylů, proto je vhodná pro využití v rámci designových systémů.

\begin{table}[ht]
    \begin{ctucolortab}
        \begin{tabularx}{\textwidth}{Y Y}
            \bfseries \textcolor{OK}{Výhody} & \bfseries \textcolor{NOT_OK}{Nevýhody} \\\Midrule{}
            Flexibilita použití              & Vyšší nároky na udržování              \\
            Rozmanitost komponent            & ---
        \end{tabularx}
    \end{ctucolortab}
    \caption{Shrnutí výhod a nevýhod knihovny React Aria}
\end{table}

\subsection{Radix UI}

Radix UI\footnote{https://radix-ui.com} je knihovna komponent, která je zároveň postavená nad Radix UI primitives.
Radix primitives využívá na pozadí headless komponent, ačkoliv by název mohl evokovat opak.

To je zásadní rozdíl oproti React Aria, kde jsou komponenty vytvořené pomocí primitives.

\begin{table}[ht]
    \begin{ctucolortab}
        \begin{tabularx}{\textwidth}{Y Y}
            \bfseries \textcolor{OK}{Výhody} & \bfseries \textcolor{NOT_OK}{Nevýhody} \\\Midrule{}
            Jednoduchost použití             & Vyšší nároky na udržování              \\
            ---                              & Nižší rozmanitost komponent
        \end{tabularx}
    \end{ctucolortab}
    \caption{Shrnutí výhod a nevýhod knihovny Radix UI}
\end{table}

\subsection{Shadcn UI}

Shadcn UI\footnote{https://ui.shadcn.com} je copy and paste knihovna.
Z dokumentace nebo \gls{cli} dostupného skrze npm je možné nakopírovat předpis komponenty do uživatelem vybraného projektu.
Hodně komponent je založeno na Radix UI primitives.

\begin{table}[ht]
    \begin{ctucolortab}
        \begin{tabularx}{\textwidth}{Y Y}
            \bfseries \textcolor{OK}{Výhody} & \bfseries \textcolor{NOT_OK}{Nevýhody}         \\\Midrule{}
            Jednoduchost použití             & Flexibilita je závislá na použitých knihovnách \\
            Možnost upravit zdrojový kód     & ---
        \end{tabularx}
    \end{ctucolortab}
    \caption{Shrnutí výhod a nevýhod Shadcn UI}
\end{table}

\section{Svelte}

Svelte je novější framework, který se ubírá cestou nejvyššího komfortu pro vývojáře.
Rozhodl jsem se provést rešerši v rámci tohoto frameworku, protože Solid.js i Svelte používají reaktivní model změn.

\subsection{Melt UI}

Podobně jako React Aria je Melt UI\footnote{\url{https://melt-ui.com}} knihovna primitiv, v kontextu této knihovny se primitiva nazývají ``builders''.
Uživatelé konzumují tyto primitiva vytváří si tak komponenty s vlastní HTML strukturou.
Z toho plyne, že je knihovna velice flexibilní na používání, ale zároveň je tak zvýšená náročnost na udržování komponent.
Výhodou je, že je možnost zde vytvořit příkladové použití těchto primitiv včetně základních stylů a usnadnit tak konzumentům práci.

\begin{table}[ht]
    \begin{ctucolortab}
        \begin{tabularx}{\textwidth}{Y Y}
            \bfseries \textcolor{OK}{Výhody} & \bfseries \textcolor{NOT_OK}{Nevýhody} \\\Midrule{}
            Vysoká flexibilita použití       & Vyšší nároky na udržování
        \end{tabularx}
    \end{ctucolortab}
    \caption{Shrnutí výhod a nevýhod knihovny Melt UI}
\end{table}

\section{Solid.js}
\label{sec:solid}

Solid.js je knihovna použitá v rámci této práce, proto je důležité provést rešerši a určit tak chybějící komponenty v rámci ekosystému.
Případně zjistit, zda lze navázat na práci již existujících knihoven.

\subsection{Solid Aria}

Solid Aria\footnote{\url{https://github.com/solidjs-community/solid-aria}} je port React Aria do Solid.js ekosystému podporovaný komunitou.
Bohužel vývoj zde již není přílis aktivní, protože narozdíl od React Aria tento projekt není sponzorovaný větší společností.
Nicméně je zde možnost navázat na práci komunity a pokračovat v rozvoji této knihovny.

Velkou výhodou je, že port React kódu do Solid.js je na hodně místech velice podobný, proto je možné využít pokrok na samotné React Aria knihovně i zde.

\begin{table}[ht]
    \begin{ctucolortab}
        \begin{tabularx}{\textwidth}{Y Y}
            \bfseries \textcolor{OK}{Výhody} & \bfseries \textcolor{NOT_OK}{Nevýhody} \\\Midrule{}
            Flexibilita použití              & Neaktivní vývoj                        \\
            ---                              & Chybějící základní komponenty
        \end{tabularx}
    \end{ctucolortab}
    \caption{Shrnutí výhod a nevýhod Solid Aria}
\end{table}

\subsection{Kobalte}

Kobalte\footnote{\url{https://kobalte.dev}} je UI toolkit pro Solid.js, který obsahuje headless komponenty.
Hodně práce na Kobalte vychází ze Solid Aria, protože minimálně jeden z hlavních vývojářů Kobalte hojně pracoval i na Solid Aria.

\begin{table}[ht]
    \begin{ctucolortab}
        \begin{tabularx}{\textwidth}{Y Y}
            \bfseries \textcolor{OK}{Výhody} & \bfseries \textcolor{NOT_OK}{Nevýhody} \\\Midrule{}
            Jednoduchost použití             & Nižší flexibilita
        \end{tabularx}
    \end{ctucolortab}
    \caption{Shrnutí výhod a nevýhod knihovny Radix UI}
\end{table}

\section{Závěr rešerše}

Z rešerše vyplynulo, že ekosystém okolo Solid.js obsahuje několik knihoven, které řeší problematiku znovupoužitelných, přístupných komponent.
Další kapitola se zaměří na analýzu chybějících komponent z \gls{apg} návrhových vzorů a zda je možné navázat na práci již existujících knihoven.

