\chapter{Implementace}

Tato kapitola popisuje implementaci dokumentace a komponent.

\section{Carousel}

TODO

\section{Disclosure}

TODO

\section{SpinButton}

TODO

\section{Toolbar}

TODO

\section{Dokumentace}

Vybranou technologií pro vytvoření dokumentace je \fr{Starlight}\footnote{\url{https://starlight.astro.build}} template postavený nad \fr{Astro}\footnote{\url{https://astro.build}} knihovnou.
Astro je statický generátor stránek, který používá vlastní syntaxi pro vytváření komponent.
Velkou výhodou je, že Astro umožňuje vnořit komponenty z jiných frameworků přímo na stránku bez použití \textit{iframe}.
Tento mechanismus lze použít pro vložení příkladové implementace komponenty přímo na stránku s dokumentací a zjednodušit tak uživatelům pochopení a účel komponenty.

Další důvod pro výběr Astro knihovny je možnost použití \fr{MDX}\footnote{\url{https://mdxjs.com}} syntaxe pro psaní dokumentace, která kombinuje markdown syntaxi s \gls{jsx}.

\section{Demo aplikace}

Důležitým aspektem pro demo aplikaci je co nejbližší simulace reálného prostředí, kde se komponenty budou používat.
Pro tento účel jsem zvolil meta-framework \fr{SolidStart}. Meta-framework je své podstatě framework postavený nad vícero frameworky, který zjednodušuje různé aspekty vývoje.
V kontextu meta-frameworků na webu se může jednat například o \textit{routing}, \textit{server-side rendering} a \textit{serverless} funkce~\cite{prismic-metaframework}.

\fr{SolidStart} je oficiálně podporovaný a vyvíjený autorem Ryanem Carniatem, který je také autorem Solid.js.
Výhoda testování komponent v prostředí meta-frameworku spočívá v tom, že komponenta je vystavená různým kontextům jako je server-side rendering a hydratace, ``client-side rendering'', nebo ``pre-rendering''.
