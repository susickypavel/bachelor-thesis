\chapter{Testování}

Tato kapitola popisuje testování implementovaných komponent.

\section{Jednotkové testy}

Jednotkové testy kontrolují logiku komponent, aby byla zaručena jejich správná implementace podle \gls{apg} návrhového vzoru.
Zvolený \textit{runner}\footnote{Software, který spouští testy a generuje výsledek.} pro jednotkové testy je \fr{vitest}\footnote{\url{https://vitest.dev}} a k tomu další pomocné knihovny:

\begin{enumerate}
    \item \fr{@vitest/coverage-v8}\footnote{\url{https://vitest.dev/guide/coverage}} --- pro generování pokrytí kódu testy.
    \item \fr{@vitest/ui}\footnote{\url{https://vitest.dev/guide/ui}} --- pro generování grafického zobrazení výsledků testů.
    \item \fr{@testing-library/*}\footnote{\url{https://testing-library.com}} --- pro testování uživatelského rozhraní.
    \item \fr{happy-dom}\footnote{\url{https://npmjs.com/package/happy-dom}} --- pro testování DOM interakcí bez prohlížeče.
\end{enumerate}

Cílem testů bylo pokrýt většinu požadavků z \gls{apg} návrhových vzorů, proto by pokrytí testů mělo být vysoké a u všech komponent přesahuje >90\% otestovaných řádků implementace\footnote{\url{https://susickypavel.github.io/solid-apg/coverage}}.

% TODO: Odkázat se na testy, kde se to testuje, abych dokázal, že ty testy opravdu existují.

Nejčastěji testovaná logika se týkala správného nastavení \gls{waiaria} atributů a klávesové navigace komponenty.
Také byly testovány různé varianty vstupních argumentů a návratových hodnot.

\section{Testování v prohlížeči}\label{chapter:browser-testing}

Nefunkční požadavek \hyperref[nfr11]{\fr{NFR 1.1}} říká, že komponenty by neměly vyhazovat výjimky ve výše definovaných prohlížečích.
Pro tento účel a nefunkční požadavek \hyperref[nfr15]{\fr{NFR 1.5}} vznikla demo aplikace, kde je možné implementované komponenty vyzkoušet\footnote{\url{https://solid-apg-app.vercel.app}}.
V budoucnu bych chtěl zautomatizovat testování v prohlížeči pomocí \gls{e2e} testovacího nástroje \fr{playwright}\footnote{\url{https://playwright.dev}}.
Z časových důvodů se \gls{e2e} testování nepodařilo implementovat v rámci této práce.

\section{Testování s odečítačem obrazovky}

Nefunkční požadavek \hyperref[nfr12]{\fr{NFR 1.2}} říká, že by komponenty měly fungovat správně s odečítačem obrazovky VoiceOver na MacOS.
Pro kontrolu, že implementace funguje správně jsem zvolil manuální testování v prohlížeči na bázi chromia\footnote{\url{https://chromium.org/chromium-projects}}.
Videa z testování jsou dostupná v dokumentaci na stránce o vybrané komponentě\footnote{\url{https://solid-apg-docs.vercel.app}}.

Je důležité podotknout, že podobně jako testování přívětivosti uživatelského prostředí je důležité brát v potaz, že uživatelům nemusí vyhovovat jakým způsobem komponenty reagují na jejich změny a to i přestože jsou implementované podle standardu \gls{waiaria}.
Vývojáři by v praxi měli využít zpětné vazby jejich uživatelů a případně upravit implementaci tak, aby vyhovovala jejich potřebám, avšak standard \gls{waiaria} vychází z dlouholetých zkoumání problematiky přístupnosti, proto komponenty splňující tento standard by měly být dostatečně přístupné pro většinu uživatelů.
