\chapter{Technologie a metodiky přístupnosti na webu}

V této kapitole se věnuji představení problematice přístupnosti na webu.

\section{Přístupný web}

Přístupnost na webu je definována jako schopnost webového obsahu být interpretován a používán nejširším možným spektrem uživatelů bez ohledu na jejich schopnosti, nebo fyzický stav~\cite{w3-accessibility}.

\section{Základní principy přístupnosti}

V této sekci popisuji základní principy přístupnosti, které jsou relevantní v kontextu této práce, tedy vytváření znovpoužitelných komponent.
Existují avšak i další principy, které se obecně zaměřují na webový obsah, porozumění textu anebo přístupnost audiovizuálního obsahu~\cite{w3-accessibility-principles}.

\subsection{Ovládání pomocí klávesnice}

Myš je základní výbavou každého uživatele na internetu, ale kromě myši máme i další periférie.
Klávesnice je jedna z nejdůležitějších, ale i mnohdy opomíjených periférií v kontextu přístupnosti.

% TODO: Focus označit, nebo dát do akronymů.

Klíčovým prvkem přístupnosti pomocí klávesnice je focus, tedy kde se uživatel zrovna nachází na dané stránce.
Společně s odečítači obrazovky tvoří základní výbavu pro nevidomé uživatele, kteří si za jejich pomocí dokáží přečíst a používat obsah na webu.

Přístupný web by měl splňovat základní pravidla pro optimální ovládaní klávesnicí~\cite{wcag-keyboard}:

\begin{itemize}
    \item Všechny funkcionality a interakce dostupné pomocí počítačové myši jsou dostupné i skrze klávesnici.
    \item Všechny funkcionality a interakce nevyžadují stisk kláves v definovaném pořadí pokud to nevyžaduje povaha vstupu.
    \item Nedochází k uváznutí focusu v jakékoliv sekci stránky.
\end{itemize}

% TODO: Udelat subsection čistě pro touch input (komplikované samo o sobě)

\subsection{Optimalizace ostatních způsobů interakce}

Kromě myši a klávesnice existují další způsoby interakce, na mobilních zařízení například dotykový displej.
Další variantou může být hlasové zadávání.
Výčet důležitých pravidel pro optimalizaci interakce za pomocí těchto způsobů~\cite{w3-accessibility-principles}:

\begin{itemize}
    \item Tlačítka, odkazy a další interaktivní prvky jsou dostatečně veliké pro dotyk prstem.
    \item Popisky interaktivních prvků jsou řádně propojeny v kódu pro korektní čtění pomocí odečítačů obrazovky a hlasovému zadávání.
\end{itemize}

\subsection{Dostatek času na interakci}

Dalším klíčovým aspektem přístupných komponent je dát uživatelům dostatek času na interakci s danou komponentou.
Přílis rychlé uzavírání, mizení, schování interaktivních prvků komponent může vést k frustraci či znemožnění použití důležité funkcionality pro uživatele~\cite{w3-accessibility-principles}.

\section{Web accessibility initiative}

\section{Accessible Rich Internet Applications}

\section{Web content accessibility guidelines}

\section{Aria authoring practices guide}

% \section{Čtečky obrazovky}
