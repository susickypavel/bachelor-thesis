\chapter{Přístupnost na webu}

V této kapitole se věnuji představení problematice přístupnosti na webu.

\section{Přístupný web}

Přístupnost na webu je definována jako schopnost webového obsahu být interpretován a používán nejširším možným spektrem uživatelů bez ohledu na jejich schopnosti, nebo fyzický stav~\cite{w3-accessibility}.

\subsection{Typy postižení}

% TODO: Elaborate on disabilities

\begin{itemize}
    \item Vizuální
    \item Motorické
    \item Sluchové
    \item Záchvaty
    \item Kognitivní
\end{itemize}

\section{W3C a WAI}

% TODO: Potřebuje poupravit slova.

\gls{w3c} je mezinárodní konsorcium, které vyvíjí standardy pro web.
\gls{wai} je jedním z projektů \gls{w3c}, který se zaměřuje na přístupnost webu.
WAI definuje specifikace a doporučení pro mnohé aspekty přístupnosti na webu, od user agentů, evaluačních nástrojů až po nástroje pro tvorbu digitálního obsahu.

\section{WAI-ARIA}

\gls{waiaria} je technická specifikace, která rožšiřuje webové technologie (HTML, JavaScript, Ajax) o ontologii attributů představující role, vlastnosti a stavy elementů~\cite{wai-aria}.
Asistivní technologie (například čtečky obrazovky) umožní uživatelům předat podrobnější informace o elementech na webu na základě těchto atributů.

\clearpage

\subsection{ARIA role}

Je důležité podotknout, že většina HTML značek má svůj sémantický význam bez použití rolí.
Například button element se používá pro interaktivní tlačítka a je vhodnější jej použít namísto nesemantického divu s rolí button.


\begin{lstlisting}[caption={Ukázka aria rolí}, label={aria-roles}, language=html]
<ul>
    <li role="menuitem">Open file</li>
    <li role="menuitem">Save file</li>
</ul>
\end{lstlisting}

\subsection{ARIA stavy}

\begin{lstlisting}[caption={Ukázka aria stavových attributů}, label={aria-states}, language=html]
<button aria-pressed="false">
    Send order
</button>
\end{lstlisting}

\subsection{ARIA vlastnosti}

\begin{lstlisting}[caption={Ukázka aria vlastností}, label={aria-properties}, language=html]
<label for="username">Username</label>
<input id="username" aria-describedby="username-error" />
<p id="username-error">Username is required</p>
\end{lstlisting}

\section{WCAG}

\section{Základní principy přístupnosti}

V této sekci popisuji základní principy přístupnosti, které jsou relevantní v kontextu této práce, tedy vytváření znovpoužitelných komponent.
Existují avšak i další principy, které se obecně zaměřují na webový obsah, porozumění textu anebo přístupnost audiovizuálního obsahu~\cite{w3-accessibility-principles}.

\subsection{Ovládání pomocí klávesnice}

Myš je základní výbavou každého uživatele na internetu, ale kromě myši máme i další periférie.
Klávesnice je jedna z nejdůležitějších, ale i mnohdy opomíjených periférií v kontextu přístupnosti.

% TODO: Focus označit, nebo dát do akronymů.

Klíčovým prvkem přístupnosti pomocí klávesnice je focus, tedy kde se uživatel zrovna nachází na dané stránce.
Společně s odečítači obrazovky tvoří základní výbavu pro nevidomé uživatele, kteří si za jejich pomocí dokáží přečíst a používat obsah na webu.

Přístupný web by měl splňovat základní pravidla pro optimální ovládaní klávesnicí~\cite{wcag-keyboard}:

\begin{itemize}
    \item Všechny funkcionality a interakce dostupné pomocí počítačové myši jsou dostupné i skrze klávesnici.
    \item Všechny funkcionality a interakce nevyžadují stisk kláves v definovaném pořadí pokud to nevyžaduje povaha vstupu.
    \item Nedochází k uváznutí focusu v jakékoliv sekci stránky.
\end{itemize}

\subsection{Ovládání pomocí dotyku}

Dalším typickým způsobem ovládání je dotykový displej na moderních zařízení typu mobil, tablet, chytré hodinky, kiosky či jiné vestavné systémy.
Ovládání za pomocí dotyku je rozmanité, uživatel může používat jednoduché dotyky, ale i složitější gesta pro interakci.

% TODO: path-based označit, nebo dát do akronymů.

Dle přednášky Iva Malého z kurzu principy mobilních aplikací rozdělujeme gesta dle počtu použitých prstů, charakteristiky pohybu a použitých technologií~\cite{ctu-pda-11}.
\gls{wcag} definuje gesta více abstraktně na tzv. path-based (kromě koncové pozice prstu záleží i na přechodných pozicích) a multipoint (gesto vyžadující více prstů)~\cite{wcag-pointer-gestures}.

Mezi základní pravidla pro optimalizaci interakce za pomocí dotyku patří:

\begin{itemize}
    \item Gesta nevyžadují přesnost, nebo je k dispozici alternativní způsob ovládání~\cite{wcag-pointer-gestures}.
    \item Interaktivní prvky jsou dostatečně velké pro dotyk prstem~\cite{wcag-target-size}.
\end{itemize}

\subsection{Ostatní způsoby interakce}

% head pointer, eye-gaze system, or speech-controlled mouse emulator.
% joystick, trackpad, graphics tablet, stylus

Kromě myši a klávesnice existují další způsoby interakce.
Další variantou může být hlasové zadávání.
Výčet důležitých pravidel pro optimalizaci interakce za pomocí těchto způsobů~\cite{w3-accessibility-principles}:

\begin{itemize}
    \item Popisky interaktivních prvků jsou řádně propojeny v kódu pro korektní čtění pomocí odečítačů obrazovky a hlasovému zadávání.
\end{itemize}

\subsection{Dostatek času na interakci}

Dalším klíčovým aspektem přístupných komponent je dát uživatelům dostatek času na interakci s danou komponentou.
Přílis rychlé uzavírání, mizení, schování interaktivních prvků komponent může vést k frustraci či znemožnění použití důležité funkcionality pro uživatele~\cite{w3-accessibility-principles}.

\section{APG}

% \section{Čtečky obrazovky}
