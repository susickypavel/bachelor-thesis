\chapter{Přístupnost na webu}

V této kapitole se věnuji představení problematice přístupnosti na webu.

\section{Přístupný web}

Přístupnost na webu je definována jako schopnost webového obsahu být interpretován a používán nejširším možným spektrem uživatelů bez ohledu na jejich schopnosti, nebo fyzický stav~\cite{w3-accessibility}.

\subsection{Typy postižení}

% TODO: Elaborate on disabilities

\begin{itemize}
  \item Vizuální
  \item Motorické
  \item Sluchové
  \item Záchvaty
  \item Kognitivní
\end{itemize}

\section{W3C a WAI}

% TODO: Potřebuje poupravit slova.

\gls{w3c} je mezinárodní konsorcium, které vyvíjí standardy pro web.
\gls{wai} je jedním z projektů \gls{w3c}, který se zaměřuje na přístupnost webu.
WAI definuje specifikace a doporučení pro mnohé aspekty přístupnosti na webu, od user agentů, evaluačních nástrojů až po nástroje pro tvorbu digitálního obsahu.

\section{WAI-ARIA}

\gls{waiaria} je technická specifikace, která rožšiřuje webové technologie (HTML, JavaScript, Ajax) o ontologii attributů představující role, vlastnosti a stavy elementů~\cite{wai-aria}.
Zmíněné atributy umožňují asistivním technologiím (například čtečky obrazovky) předat uživatelům význam a podrobnější informace o elementech, které se nachází na daném webu a tím jim pomoci s pochopením a manipulací obsahu stránky.

\subsection{ARIA role}

Role slouží k přidání sémantického významu HTML elementům tak, aby nevidomí či uživatelé s jinou formou postižení dokázali pochopit účel daného elementu.

Některé HTML elementy mohou mít svůj sémantický význam i bez použití rolí~\cite{wai-implicit-semantics}.
Například button element se používá pro interaktivní tlačítka a je vhodnější jej použít namísto nesemantického divu s rolí button.

\begin{lstlisting}[caption={Aria role}, label={aria-roles}, language=html]
<ul role="menu">(*@\label{aria-roles-1}@*)
  <li role="menuitem">Open file</li>(*@\label{aria-roles-2}@*)
  <li role="menuitem">Save file</li>(*@\label{aria-roles-3}@*)
</ul>
\end{lstlisting}

V ukázce~\ref{aria-roles} na řádku~\ref{aria-roles-1} můžeme vidět použití role menu na elementu pro neuspořádaný seznam.
Na řádku~\ref{aria-roles-2} a~\ref{aria-roles-3} máme dva prvky s rolí menuitem, které reprezentují položky v menu.
Každá role kromě sémantikého významu s sebou nese i vlastnosti, které musí být dodrženy podle specifikace.
Napříkad výše zmíněné menu musí obsahovat minimálně jeden prven s rolí menuitem (alternativně menuitemcheckbox nebo menuitemradio)~\cite{wai-required-owned-elements,wai-standard-guidelines-required-owned-elements}.

Pro tvorbu moderních webových komponent je vhodné preferovat sémantické elementy, které mají svůj význam i bez použití explicitních rolí.
Explicitně definované role jsou potřeba pokud informace o elementu nejsou dostatečné pro pochopení jeho účelu.

V kontextu práce budu používat především role z kategorie widget podle \gls{waiaria} specifikace.
Existují však i další kategorie rolí určené pro strukturu webu nebo orientační body~\cite{wai-catorization-of-roles}.

\clearpage

\subsection{ARIA stavy}

Stavy slouží pro upozornění uživatele na změny ve stavu stránky nebo komponenty.
Společně s rolemi tvoří základní dvojici pro přístupnost na webu.
Je důležité poznamenat, že některé stavy nejdou použít pro libovolnou roli, ale jsou vázané ke specifickým účelům a rolím.

\begin{lstlisting}[caption={Aria stavové atributy}, label={aria-states}, language=html]
<button aria-pressed="false">(*@\label{aria-states-1}@*) <!-- true|false|mixed -->
  Send order(*@\label{aria-states-2}@*)
</button>(*@\label{aria-states-3}@*)
\end{lstlisting}

V ukázce~\ref{aria-states} na řádku~\ref{aria-states-1} můžeme vidět použití attributu aria-pressed.
Tento atribut slouží jenom pro elementy s rolí button.
Řádek~\ref{aria-states-1} nám zároveň ukazuje, že hodnota atributu je false, ale může nabývat i hodnot true nebo mixed.
Button s takovým atributem představuje tzv. toggle button~\cite{mdn-aria-pressed}. Tedy tlačítko, které si drží svůj vlastní stav zakliknutí.
Můžeme využít JavaScriptu pro dynamické změny atributu na reakci uživatelova vstupu.

\subsection{ARIA vlastnosti}

Vlastnosti jsou podobné stavům.
Jedná se o HTML atributy jejichž primárním účelem je dodání více informací o daném elementu.
Rozdíl oproti stavům vychází z četnosti změn, kde u stavů se předpokládá částá změna v životním cyklu stránky, která vede na upozornění uživatele.
Oproti tomu vlastnosti se mění jenom zřídka.

\begin{lstlisting}[caption={Aria vlastnosti}, label={aria-properties}, language=html]
<label for="username">Username</label>(*@\label{aria-properties-1}@*)
<input id="username" aria-describedby="username-error" />(*@\label{aria-properties-2}@*)
<p id="username-error">Username is required</p>(*@\label{aria-properties-3}@*)
\end{lstlisting}

V ukázce~\ref{aria-properties} na řádku~\ref{aria-properties-2} můžeme vidět použití vlastnosti aria-describedby.
Tento atribut slouží k přidání dodatečnému popisu daného input elementu. Hodnota atributu je id elementu, který obsahuje textový popis daného inputu.

\section{WCAG}

\gls{wcag} je soubor pravidel pro tvorbu přístupných webů.
Narozdíl od specifikace \gls{waiaria} se \gls{wcag} zaměřuje na pravidla, jak by se stránky měly chovat, aby byly pro uživatele co nejpřívětivější na používání.
Nejedná se tak o technickou specifikaci, která by tvořila nové mechanismy pro přístupnost.
Všechny doporučení jsou dosažitelné pomocí existujících \gls{waiaria} atributů a JavaScriptu.

\subsection{Úrovně přístupnosti}

TODO

\section{Základní principy přístupnosti}

V této sekci popisuji základní principy přístupnosti, které jsou relevantní v kontextu této práce, tedy vytváření znovupoužitelných komponent.
Existují avšak i další principy, které se obecně zaměřují na webový obsah, porozumění textu anebo přístupnost audiovizuálního obsahu~\cite{w3-accessibility-principles}.

\subsection{Ovládání pomocí klávesnice}

Myš je základní výbavou každého uživatele na internetu, ale kromě myši máme i další periférie.
Klávesnice je jedna z nejdůležitějších, ale i mnohdy opomíjených periférií v kontextu přístupnosti.

% TODO: Focus označit, nebo dát do akronymů.

Klíčovým prvkem přístupnosti pomocí klávesnice je focus, tedy kde se uživatel zrovna nachází na dané stránce.
Společně s odečítači obrazovky tvoří základní výbavu pro nevidomé uživatele, kteří si za jejich pomocí dokáží přečíst a používat obsah na webu.

Přístupný web by měl splňovat základní pravidla pro optimální ovládaní klávesnicí~\cite{wcag-keyboard}:

\begin{itemize}
  \item Všechny funkcionality a interakce dostupné pomocí počítačové myši jsou dostupné i skrze klávesnici.
  \item Všechny funkcionality a interakce nevyžadují stisk kláves v definovaném pořadí pokud to nevyžaduje povaha vstupu.
  \item Nedochází k uváznutí focusu v jakékoliv sekci stránky.
\end{itemize}

\subsection{Ovládání pomocí dotyku}

Dalším typickým způsobem ovládání je dotykový displej na moderních zařízení typu mobil, tablet, chytré hodinky, kiosky či jiné vestavné systémy.
Ovládání za pomocí dotyku je rozmanité, uživatel může používat jednoduché dotyky, ale i složitější gesta pro interakci.

% TODO: path-based označit, nebo dát do akronymů.

Dle přednášky Iva Malého z kurzu principy mobilních aplikací rozdělujeme gesta dle počtu použitých prstů, charakteristiky pohybu a použitých technologií~\cite{ctu-pda-11}.
\gls{wcag} definuje gesta více abstraktně na tzv. path-based (kromě koncové pozice prstu záleží i na přechodných pozicích) a multipoint (gesto vyžadující více prstů)~\cite{wcag-pointer-gestures}.

Mezi základní pravidla pro optimalizaci interakce za pomocí dotyku patří:

\begin{itemize}
  \item Gesta nevyžadují přesnost, nebo je k dispozici alternativní způsob ovládání~\cite{wcag-pointer-gestures}.
  \item Interaktivní prvky jsou dostatečně velké pro dotyk prstem~\cite{wcag-target-size}.
\end{itemize}

\subsection{Ostatní způsoby interakce}

% head pointer, eye-gaze system, or speech-controlled mouse emulator.
% joystick, trackpad, graphics tablet, stylus

Kromě myši a klávesnice existují další způsoby interakce.
Další variantou může být hlasové zadávání.
Výčet důležitých pravidel pro optimalizaci interakce za pomocí těchto způsobů~\cite{w3-accessibility-principles}:

\begin{itemize}
  \item Popisky interaktivních prvků jsou řádně propojeny v kódu pro korektní čtění pomocí odečítačů obrazovky a hlasovému zadávání.
\end{itemize}

\subsection{Dostatek času na interakci}

Dalším klíčovým aspektem přístupných komponent je dát uživatelům dostatek času na interakci s danou komponentou.
Přílis rychlé uzavírání, mizení, schování interaktivních prvků komponent může vést k frustraci či znemožnění použití důležité funkcionality pro uživatele~\cite{w3-accessibility-principles}.

\section{APG}

% \section{Čtečky obrazovky}
