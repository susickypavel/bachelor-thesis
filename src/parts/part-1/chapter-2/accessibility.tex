\chapter{Technologie a metodiky přístupnosti na webu}

V této kapitole se věnuji představení problematice přístupnosti na webu.

\section{Přístupný web}

Přístupnost na webu je definována jako schopnost webového obsahu být interpretován a používán nejširším možným spektrem uživatelů bez ohledu na jejich schopnosti, nebo fyzický stav~\cite{w3-accessibility}.

\section{Základní principy přístupnosti}

V této sekci popisuji základní principy přístupnosti, které jsou relevantní v kontextu této práce, tedy vytváření znovpoužitelných komponent.
Existují avšak i další principy, které se obecně zaměřují na webový obsah, porozumění textu anebo přístupnost audiovizuálního obsahu~\cite{w3-accessibility-principles}.

\subsection{Ovládání pomocí klávesnice}

Myš je základní výbavou každého uživatele na internetu, ale kromě myši máme i další periférie.
Klávesnice je jedna z nejdůležitějších, ale i mnohdy opomíjených periférií v kontextu přístupnosti.

% TODO: Focus označit, nebo dát do akronymů.

Klíčovým prvkem přístupnosti pomocí klávesnice je focus, tedy kde se uživatel zrovna nachází na dané stránce.
Společně s odečítači obrazovky tvoří základní výbavu pro nevidomé uživatele, kteří si za jejich pomocí dokáží přečíst a používat obsah na webu.

Přístupný web by měl splňovat základní pravidla pro optimální ovládaní klávesnicí~\cite{wcag-keyboard}:

\begin{itemize}
    \item Všechny funkcionality a interakce dostupné pomocí počítačové myši jsou dostupné i skrze klávesnici.
    \item Všechny funkcionality a interakce nevyžadují stisk kláves v definovaném pořadí pokud to nevyžaduje povaha vstupu.
    \item Nedochází k uvíznutí focusu v jakékoliv sekci stránky.
\end{itemize}


\section{Web accessibility initiative}

\section{Accessible Rich Internet Applications}

\section{Web content accessibility guidelines}

\section{Aria authoring practices guide}

% \section{Čtečky obrazovky}
