\chapter{Technologie}

V této kapitoly jsou rozebrány technologie použité pro tuto práci.

\section{Svelte}\label{sec:Svelte}

Svelte je reaktivní framework pro tvorbu uživatelských rozhraní obdobně jako populární alternativy React, nebo Vue.
Největší rozdíl oproti těmto frameworkům se vyskytuje v tom, že Svelte neobsahuje runtime jako tomu je právě v Reactu.
Při kompilaci se komponenty a reaktivní deklarace převedou na imperativní kód v JavaScriptu, který obsluhuje změny v \gls{dom}.

\subsection{Výhody}

Mezi hlavní přednosti Svelte oproti jiným frameworkům patří:

\begin{itemize}
    \item \textbf{Vývojářská přívetivost (DX)} --- Svelte je v praxi jednoduché a intuitivní i pro nové vývojáře.
\end{itemize}

\begin{table}[ht]
    \begin{ctucolortab}
        \begin{tabular}{c|c|c|c}
            \bfseries Metrika & \bfseries Svelte & \bfseries Vue & \bfseries React \\\Midrule{}
            Operations        & \textbf{1.29}    & 1.32          & 1.66            \\
            Startup metrics   & \textbf{1.02}    & 1.26          & 1.66            \\
            Memory allocation & \textbf{1.46}    & 1.86          & 2.58
        \end{tabular}
    \end{ctucolortab}
    \caption{Porovnání rychlosti Svelte s populárními frameworky}
    \label{tab:foobar}
\end{table}

\clearpage

\subsection{Virtual DOM vs Reaktivita}

Důležitý rozdíl mezi Svelte a podobnými frameworky je ten, že neobsahuje \gls{dom} diffing algoritmus jako to je u Reactu, nebo Vue.
Veškeré změny v \gls{dom} jsou ve Svelte řešené pomocí reaktivních proměnných, které automaticky při své změně vyvolají změnu i ve zmíněném \gls{dom}.
Sledování změn je zaručeno na základě vytvořeného imperativního kódu při kompilaci (viz. sekce~\ref{sec:Svelte} a ukázky kódu~\ref{svelte-counter} a~\ref{svelte-counter-compiled}).

\begin{lstlisting}[caption={Počítadlo ve Svelte}, label={svelte-counter}, language=html]
<script>
	let count = 0
	
	function add() {
		count++
	}
</script>

<button on:click={add}>+1</button>
<p>Current count: {count}</p>
\end{lstlisting}

\begin{lstlisting}[caption={Počítadlo po kompilaci}, label={svelte-counter-compiled}, language=JavaScript]
function instance($$self, $$props, $$invalidate) {
    let count = 0;

    function add() {
        $$invalidate(0, count++, count);
    }

    return [count, add];
}

class App extends SvelteComponent {
    constructor(options) {
        super();
        init(this, options, instance, create_fragment, safe_not_equal, {});
    }
}
\end{lstlisting}

\subsection{Svelte 5.0}
