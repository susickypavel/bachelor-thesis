\chapter{Technologie}

V této kapitolě jsou rozebrány technologie použité pro tuto práci.

\section{Svelte}\label{sec:Svelte}

Svelte je kompilátor pro tvorbu uživatelských rozhraní obdobně jako populární alternativy React, nebo Vue.

% Největší rozdíl oproti těmto frameworkům se vyskytuje v tom, že Svelte neobsahuje runtime jako tomu je právě v Reactu.
% Při kompilaci se komponenty a reaktivní deklarace převedou na imperativní kód v JavaScriptu, který obsluhuje změny v \gls{dom}.

\subsection{Rozdíly}

Mezi hlavní rozdíly Svelte oproti jiným frameworkům patří:

\begin{itemize}
    \item \textbf{Kompilátor} --- Největší rozdíl oproti zmíněným frameworkům se vyskytuje v tom, že Svelte není runtime, který se posíla na klienta společně se zdrojovým kódem komponent.
          Svelte je kompilátor souborů s příponou ``.svelte'', který převede komponenty na optimalizovaný imperativní kód v JavaScriptu.
    \item \textbf{Reaktivita} --- Svelte používá reaktivní model pro změny v \gls{dom} namísto virtuálního \gls{dom}.
    \item \textbf{Podpora komunity} --- Svelte není produktem velkých společností jako je React od Meta Platforms, nebo Angular od Google.
          Zároveň už to není čistě produkt autora Riche Harrise a komunity, ale od roku 2021 je vývoj plně sponzorován platformou Vercel.
    \item \textbf{Vývojářská přívetivost (DX)} --- Svelte je v praxi jednodušší na používání a intuitivní i pro nové vývojáře.
\end{itemize}

Kompilátor s sebou nese jednu nevýhodu a to je velikost výsledného kódu po kompilaci.
Prázdný projekt ve Svelte má minimální velikost, protože zde není potřeba téměř žádného imperativního kódu pro zaručení reaktivity.
To vede k tomu, že každá nová komponenta přidává unikátní kus kódu a zvyšuje tak velikost dat, které se posílají přes internet na klienta.
Existuje tak inflexní bod, kdy velikost aplikace bude vyšší než aplikace napsané v Reactu.
V praxi se však ukazuje, že toto může být potenciálně problémové pouze u velkých aplikací s velkým počtem komponent~\cite{svelte-scaling}.

\subsection{Rychlost}

TODO

% TODO: Different background color for table

\begin{table}[ht]
    % \begin{ctucolortab}
    \begin{tabular}{c|c|c|c|c}
        \bfseries Metrika & \bfseries{Svelte 5} & \bfseries{Svelte 4} & \bfseries{Vue} & \bfseries{React} \\\Midrule{}
        Operations        & 1.06                & \textbf{1.29}       & 1.32           & 1.66             \\
        Startup metrics   & 1.06                & \textbf{1.02}       & 1.26           & 1.66             \\
        Memory allocation & 1.48                & \textbf{1.46}       & 1.86           & 2.58
    \end{tabular}
    % \end{ctucolortab}
    \caption{Porovnání rychlosti Svelte s populárními frameworky}
    \label{tab:foobar}
\end{table}

\clearpage

\subsection{Virtual DOM vs Reaktivita}

Důležitý rozdíl mezi Svelte a podobnými frameworky je ten, že neobsahuje \gls{dom} diffing algoritmus jako to je u Reactu.
Veškeré změny v \gls{dom} jsou ve Svelte řešené pomocí reaktivních proměnných, které automaticky při své změně vyvolají změnu i ve zmíněném \gls{dom}.
Sledování změn je zaručeno na základě vytvořeného imperativního kódu při kompilaci (viz. sekce~\ref{sec:Svelte} a ukázky kódu~\ref{svelte-counter} a~\ref{svelte-counter-compiled}).

\begin{lstlisting}[caption={Počítadlo ve Svelte 4}, label={svelte-counter}, language=html]
<script>
	let count = 0
	
	function add() {
		count++
	}
</script>

<button on:click={add}>+1</button>
<p>Current count: {count}</p>
\end{lstlisting}

\begin{lstlisting}[caption={Počítadlo po kompilaci}, label={svelte-counter-compiled}, language=JavaScript]
function instance($$self, $$props, $$invalidate) {
    let count = 0;

    function add() {
        $$invalidate(0, count++, count);
    }

    return [count, add];
}

class App extends SvelteComponent {
    constructor(options) {
        super();
        init(this, options, instance, create_fragment, safe_not_equal, {});
    }
}
\end{lstlisting}

\subsection{Svelte 5.0}
